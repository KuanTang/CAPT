\documentclass[a4paper, 12pt]{article}

\usepackage{geometry}



\usepackage{subcaption}%标题
\usepackage{authblk}


\usepackage[T1]{fontenc}

\usepackage{setspace}
\renewcommand{\baselinestretch}{1.0}


\usepackage{indentfirst}
\usepackage{booktabs}
\usepackage{multirow}
\usepackage{float}
\usepackage{linguex}
\usepackage{graphicx}


\usepackage{booktabs}
\usepackage{tikz}
\usepackage[titletoc]{appendix}
\usepackage{sidenotes}

\usepackage{url}

\usepackage[colorlinks,
            linkcolor=black,
            anchorcolor=blue,
            citecolor=blue
            ]{hyperref}

\usepackage{bbding}
\usepackage{lscape}
\usepackage{subfigure}
\usepackage{dingbat}

%\usepackage{ulem}

\usepackage{apacite}
\bibliographystyle{apacite}


\setlength{\parskip}{0.4em}

\usepackage{fancyhdr}%导入fancyhdr包
%\usepackage{ctex}%导入ctex包
%\pagenumbering{Alph}%设置页码格式,大写字母标页
\pagestyle{fancy}
\fancyhead[L]{Acoustic Phonetics
WS 2020/21}
%\fancyhead[R]{Kuan TANG}

\usepackage{qtree}
\usepackage{tree-dvips}


\title{Wang \& Lee (2013): Supervised Detection and Unsupervised Discovery
of Pronunciation Error Patterns for
Computer-Assisted Language Learning}
\author{Jingwen LI and Kuan TANG}

\affil[]{

ISCL, Seminar für Sprachwissenschaft\\
Eberhard Karls Universität Tübingen\\
\it{jingwen.li/kuan.tang@student.uni-tuebingen.de}}

\date

\begin{document}

\maketitle




\tableofcontents



\clearpage
\section{Introduction}


The aim of this paper is to introduce the work of  \citeA{wang2015supervised}. The content and authorship  of this paper as follows: In the section one, Kuan summarizes the whole paper \citeA{wang2015supervised} in general. In section 2, Jingwen introduces the new frameworks for both Supervised EP detection and  Unsupervised EP discovery. In section 3, Kuan gives more details 


\citeA{wang2015supervised} propose new frameworks for both Supervised Error Patterns Detection and Unsupervised EP Discovery \footnote{Web link \url{https://www.google.com}}.



\section{Theoretical Framework}


\subsection{Supervised EP Detection}

\subsection{Unsupervised EP Discovery}




\section{Methods}
\subsection{Cascaded adaptation techniques}
\subsection{Metrics}
Distance Measures
\subsection{Clustering}




\section{Post-reading Questions}









\bibliography{Bib}
\end{document}
